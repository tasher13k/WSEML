%! Author = dlebedin
%! Date = 31.01.2022

\documentclass{report}
\usepackage[utf8]{inputenc}
\usepackage[english,russian]{babel}
\setlength{\textheight}{27truecm}
\setlength{\textwidth}{19truecm}
\setlength{\oddsidemargin}{-1.5truecm}
\setlength{\topmargin}{-3truecm}
\begin{document}

    Обработка человеко-читаемых указателей.

    Под указателем понимается список типа ptr, представляющий собой последовательность уточняющих шагов. Первый из них называется базовым и имеет роль base. Он может иметь один из следующих типов.

    a (абсолютный адрес, указатель в компактной форме) --- может указывать на объект (в этом случае, это список из одной пары с ключом addr, данные в которой --- целое число, равное адресу указываемого объекта). Может указывать на конкретный байт в составе строки (в этом случае, к паре с адресом строки, содержащей нужный байт, добавляется еще одна с ключом offset, данными будет индекс нужного байта).

    s (текущий стек) --- указывает на объект, служащий текущим стеком, в котором находится текущий стековый кадр (актуально только при запуске программы).

    f (текущий стековый кадр) --- указывает на объект, служащий текущим кадром, находящимся в текущем стеке (актуально только при запуске программы).

    с (текущая команда) --- указывает на объект, в котором записана текущая команда (актуально только при запуске программы).

    r (корневой объект) --- указывает на весь объект.

   Все остальные шаги указателя имеют роль ps и бывают следующих типов:

    \{[i:индекс]i, \} --- такой уточняющий шаг указателя называется <<по индексу>>; в этом случае индекс должен быть целым числом (может быть отрицательным, что означает отсчет элементов с конца списка).

    \{[k:ключ]k, \} --- такой уточняющий шаг указателя называется <<по ключу>> (применять такой шаг можно только к спискам); ключ может быть любым объектом.

    \{[]u\} --- переход на один шаг наверх, к объемлющему списку.

    \{[o:сдвиг]b, \} --- переход к брату (параметром будет число, добавка в индекс, может быть как положительной, так и отрицательной).

    Задание 4.

    а) Написать функцию calc, принимающую любой указатель и возвращающую базовый шаг указателя типа a, указывающий на тот же объект.

    б) Написать функцию expand, принимающую базовый шаг указателя типа a и возвращающую указатель, начинающийся с базового шага указателя типа r, на тот же объект.

    в) Написать функцию reduce, принимающую указатель и возвращающую результат его упрощения (например, если есть подряд два уточняющих шага --- по индексу и наверх, их обоих можно просто выкинуть; аналогично, если есть несколько рядом стоящих шагов типа переход к брату, их можно объединить).

    г) Написать функцию to\_i, принимающую указатель и заменяющую все имеющиеся шаги типа по ключу на аналогичные, т. е. указывающие на те же элементы, по индексу.

    д) Написать функцию to\_k, принимающую указатель и заменяющую все имеющиеся шаги типа по индексу на аналогичные по ключу.

\end{document}
