%! Author = dlebedin
%! Date = 31.01.2022

\documentclass{report}
\usepackage[utf8]{inputenc}
\usepackage[english,russian]{babel}
\setlength{\textheight}{27truecm}
\setlength{\textwidth}{19truecm}
\setlength{\oddsidemargin}{-1.5truecm}
\setlength{\topmargin}{-3truecm}
\begin{document}

    Класс WSEML, используемый для представления программ, данных и промежуточных состояний.

    Объектами данного класса могут быть строки (последовательные наборы байтов в памяти; частными случаями являются целые и вещественные числа, а также идентификаторы) и списки (последовательности пар, их порядок следования тоже важен); кроме того, имеется один специальный объект, называемый нулевым.

    Каждый объект, кроме нулевого, имеет тип (другой объект) и указатель на ту пару, элементом которой является(если такой нет --- нулевой указатель).

    Каждая пара состоит из ключа и данных (другие объекты), для каждого из которых хранится роль (может быть любым объектом), и каждая пара хранит указатель на тот список, в состав которого она входит (сама по себе пара объектом не является, поэтому у нее нет типа, она обязательно входит в состав ровно одного списка, и не может использоваться как тип, роль, ключ или данные).

    Задание 1. Написать набор классов, реализующий вышеуказанные принципы. Срок сдачи --- примерно через неделю.

\end{document}
