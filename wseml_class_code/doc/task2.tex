%! Author = dlebedin
%! Date = 31.01.2022

\documentclass{report}
\usepackage[utf8]{inputenc}
\usepackage[english,russian]{babel}
\setlength{\textheight}{27truecm}
\setlength{\textwidth}{19truecm}
\setlength{\oddsidemargin}{-1.5truecm}
\setlength{\topmargin}{-3truecm}
\begin{document}

    Ввод объектов WSEML из строки/потока.

    Требуется написать функцию, превращающую строку в объект класса WSEML, по следующим правилам.

    Символ \$ обозначает нулевой объект.

    Строка может иметь несколько разных текстовых представлений:

    а) Последовательность символов в одинарных разных кавычках --- открывающая на той же клавише, что и тильда, закрывающая --- на той же, что и двойная кавычка. В тексте строки открывающие и закрывающие конструкции (кавычки и разного рода скобки) должны быть сбалансированы, кроме экранированных при помощи обратной косой черты.

    б) Последовательность байтов (двоичная строка) --- последовательность из четного числа шестнадцатеричных цифр в двойных кавычках; внутри везде допустимы пробельные символы, они игнорируются.

    в) Целые, рациональные (как обычные дроби и периодические десятичные дроби) и вещественные числа произвольной точности воспринимаются как строки, просто кавычки вокруг них можно не писать.

    г) Аналогично идентификаторы.

    Список представлен последовательностью пар через запятую, заключенной в фигурные скобки.

    Пара представляется как два объекта через двоеточие. У каждого из них допустимо указывать тип и роль при помощи квадратных скобок (роль[объект]тип). При этом, если объект --- список, то фигурные скобки, вложенные в квадратные, можно не писать.

    Также можно использовать конструкцию <строка, означающую строку-содержимое файла, имя которого представлено строкой-параметром.

    Конструкция \#строка означает объект, полученный из строки-параметра. Эта конструкция может сочетаться с конструкцией чтения файла, указанной в предыдущем абзаце. Важно, что в строке-параметре всегда должно содержаться текстовое представление объекта; например, просто пара или набор пар через запятую, вне списка, не допускается.

    Наконец, можно использовать круглые скобки для изменения порядка применения операций.

    Примеры.

    1. \$ --- пустой объект.

    2. `small string' --- строка

    3. ''ab cd'' --- двоичная строка из двух байтов с содержимым 0xab и 0xcd (пробел в середине игнорируется).

    4. 137 --- целое, 1.37e2 --- вещественное, 1/3 и 0.(3) --- рациональные числа.

    5. hello --- идентификатор

    6. \{1:2, 3:4\} --- список из двух пар, с ключами 1 и 3, и данными 2 и 4.

    7. \{f[1]\$:2, 3:s[`data']t\} --- список из двух пар, у ключа 1 первой из них указана роль f и тип \$, у данных `data' второй --- роль s и тип t.

    8. \{f[1:7]\$:2, 3:s[`data']t\} --- то же самое, только теперь ключ первой пары --- список из одной пары, с данными 7 и ключом 1.

    9. \{f[1:9, `df':45]\$:2, 3:s[`data']t\} --- то же самое, только теперь ключ первой пары --- список из двух пар.

    10. \{<`a-file.txt':24, 23:ident\} --- список из двух пар, ключом первой из которых будет строка, в которой находится содержимое файла a-file.txt .

    11. \{\#`\{1:2, 3:4\}':7, fbc:ert\} --- список из двух пар, ключом первой из которых будет список из двух пар, с ключами 1 и 3, и данными 2 и 4.

    12. \{\#<`a-file.txt':24, 23:ident\} --- список из двух пар, ключом первой из которых будет объект, текстовое представление которого содержится в файле a-file.txt .

    Задание 2. Написать в классе WSEML конструктор по умолчанию (NULLOBJECT), конструкторы копирования и перемещения, копирующую и перемещающую операции присваивания, деструктор.

    Задание 3. Написать функции WSEML parse(std::string), возвращающую объект, текстовое представление которого содержится в строке-параметре, и std::string pack(WSEML), возвращающую текстовое представление объекта-параметра.

\end{document}
